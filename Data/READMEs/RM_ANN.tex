\documentclass[]{article}
\usepackage{lmodern}
\usepackage{amssymb,amsmath}
\usepackage{ifxetex,ifluatex}
\usepackage{fixltx2e} % provides \textsubscript
\ifnum 0\ifxetex 1\fi\ifluatex 1\fi=0 % if pdftex
  \usepackage[T1]{fontenc}
  \usepackage[utf8]{inputenc}
\else % if luatex or xelatex
  \ifxetex
    \usepackage{mathspec}
  \else
    \usepackage{fontspec}
  \fi
  \defaultfontfeatures{Ligatures=TeX,Scale=MatchLowercase}
\fi
% use upquote if available, for straight quotes in verbatim environments
\IfFileExists{upquote.sty}{\usepackage{upquote}}{}
% use microtype if available
\IfFileExists{microtype.sty}{%
\usepackage{microtype}
\UseMicrotypeSet[protrusion]{basicmath} % disable protrusion for tt fonts
}{}
\usepackage[margin=1in]{geometry}
\usepackage{hyperref}
\hypersetup{unicode=true,
            pdftitle={README for Artificial Neural Network (ANN)},
            pdfauthor={Group 2},
            pdfborder={0 0 0},
            breaklinks=true}
\urlstyle{same}  % don't use monospace font for urls
\usepackage{color}
\usepackage{fancyvrb}
\newcommand{\VerbBar}{|}
\newcommand{\VERB}{\Verb[commandchars=\\\{\}]}
\DefineVerbatimEnvironment{Highlighting}{Verbatim}{commandchars=\\\{\}}
% Add ',fontsize=\small' for more characters per line
\usepackage{framed}
\definecolor{shadecolor}{RGB}{248,248,248}
\newenvironment{Shaded}{\begin{snugshade}}{\end{snugshade}}
\newcommand{\AlertTok}[1]{\textcolor[rgb]{0.94,0.16,0.16}{#1}}
\newcommand{\AnnotationTok}[1]{\textcolor[rgb]{0.56,0.35,0.01}{\textbf{\textit{#1}}}}
\newcommand{\AttributeTok}[1]{\textcolor[rgb]{0.77,0.63,0.00}{#1}}
\newcommand{\BaseNTok}[1]{\textcolor[rgb]{0.00,0.00,0.81}{#1}}
\newcommand{\BuiltInTok}[1]{#1}
\newcommand{\CharTok}[1]{\textcolor[rgb]{0.31,0.60,0.02}{#1}}
\newcommand{\CommentTok}[1]{\textcolor[rgb]{0.56,0.35,0.01}{\textit{#1}}}
\newcommand{\CommentVarTok}[1]{\textcolor[rgb]{0.56,0.35,0.01}{\textbf{\textit{#1}}}}
\newcommand{\ConstantTok}[1]{\textcolor[rgb]{0.00,0.00,0.00}{#1}}
\newcommand{\ControlFlowTok}[1]{\textcolor[rgb]{0.13,0.29,0.53}{\textbf{#1}}}
\newcommand{\DataTypeTok}[1]{\textcolor[rgb]{0.13,0.29,0.53}{#1}}
\newcommand{\DecValTok}[1]{\textcolor[rgb]{0.00,0.00,0.81}{#1}}
\newcommand{\DocumentationTok}[1]{\textcolor[rgb]{0.56,0.35,0.01}{\textbf{\textit{#1}}}}
\newcommand{\ErrorTok}[1]{\textcolor[rgb]{0.64,0.00,0.00}{\textbf{#1}}}
\newcommand{\ExtensionTok}[1]{#1}
\newcommand{\FloatTok}[1]{\textcolor[rgb]{0.00,0.00,0.81}{#1}}
\newcommand{\FunctionTok}[1]{\textcolor[rgb]{0.00,0.00,0.00}{#1}}
\newcommand{\ImportTok}[1]{#1}
\newcommand{\InformationTok}[1]{\textcolor[rgb]{0.56,0.35,0.01}{\textbf{\textit{#1}}}}
\newcommand{\KeywordTok}[1]{\textcolor[rgb]{0.13,0.29,0.53}{\textbf{#1}}}
\newcommand{\NormalTok}[1]{#1}
\newcommand{\OperatorTok}[1]{\textcolor[rgb]{0.81,0.36,0.00}{\textbf{#1}}}
\newcommand{\OtherTok}[1]{\textcolor[rgb]{0.56,0.35,0.01}{#1}}
\newcommand{\PreprocessorTok}[1]{\textcolor[rgb]{0.56,0.35,0.01}{\textit{#1}}}
\newcommand{\RegionMarkerTok}[1]{#1}
\newcommand{\SpecialCharTok}[1]{\textcolor[rgb]{0.00,0.00,0.00}{#1}}
\newcommand{\SpecialStringTok}[1]{\textcolor[rgb]{0.31,0.60,0.02}{#1}}
\newcommand{\StringTok}[1]{\textcolor[rgb]{0.31,0.60,0.02}{#1}}
\newcommand{\VariableTok}[1]{\textcolor[rgb]{0.00,0.00,0.00}{#1}}
\newcommand{\VerbatimStringTok}[1]{\textcolor[rgb]{0.31,0.60,0.02}{#1}}
\newcommand{\WarningTok}[1]{\textcolor[rgb]{0.56,0.35,0.01}{\textbf{\textit{#1}}}}
\usepackage{longtable,booktabs}
\usepackage{graphicx,grffile}
\makeatletter
\def\maxwidth{\ifdim\Gin@nat@width>\linewidth\linewidth\else\Gin@nat@width\fi}
\def\maxheight{\ifdim\Gin@nat@height>\textheight\textheight\else\Gin@nat@height\fi}
\makeatother
% Scale images if necessary, so that they will not overflow the page
% margins by default, and it is still possible to overwrite the defaults
% using explicit options in \includegraphics[width, height, ...]{}
\setkeys{Gin}{width=\maxwidth,height=\maxheight,keepaspectratio}
\IfFileExists{parskip.sty}{%
\usepackage{parskip}
}{% else
\setlength{\parindent}{0pt}
\setlength{\parskip}{6pt plus 2pt minus 1pt}
}
\setlength{\emergencystretch}{3em}  % prevent overfull lines
\providecommand{\tightlist}{%
  \setlength{\itemsep}{0pt}\setlength{\parskip}{0pt}}
\setcounter{secnumdepth}{0}
% Redefines (sub)paragraphs to behave more like sections
\ifx\paragraph\undefined\else
\let\oldparagraph\paragraph
\renewcommand{\paragraph}[1]{\oldparagraph{#1}\mbox{}}
\fi
\ifx\subparagraph\undefined\else
\let\oldsubparagraph\subparagraph
\renewcommand{\subparagraph}[1]{\oldsubparagraph{#1}\mbox{}}
\fi

%%% Use protect on footnotes to avoid problems with footnotes in titles
\let\rmarkdownfootnote\footnote%
\def\footnote{\protect\rmarkdownfootnote}

%%% Change title format to be more compact
\usepackage{titling}

% Create subtitle command for use in maketitle
\providecommand{\subtitle}[1]{
  \posttitle{
    \begin{center}\large#1\end{center}
    }
}

\setlength{\droptitle}{-2em}

  \title{README for Artificial Neural Network (ANN)}
    \pretitle{\vspace{\droptitle}\centering\huge}
  \posttitle{\par}
    \author{Group 2}
    \preauthor{\centering\large\emph}
  \postauthor{\par}
      \predate{\centering\large\emph}
  \postdate{\par}
    \date{02 12 2019}


\begin{document}
\maketitle

\textbf{Corresponding R Script:}
\href{https://github.com/NicSchuler/DSF_NFLDraftPrediction/blob/master/Project_Scripts/ANN.R}{ANN}

\hypertarget{introduction}{%
\section{1. Introduction}\label{introduction}}

The goal of this project is to make a prediction about the likelilhood
of College Football (CFB) players to be drafted into the professional
football league (NFL). In order to achieve this, different aspects of a
CFB player's college career are being used as features
\(p_{1}, ..., p_{n}\). These include game statistics such as rush
attempts \texttt{Rush.Att} or yards ran after a pass \texttt{Pass.Yard}.
To solve this problem, one of the methods employed is an artificial
neural network (ANN). ANNs consist of connected nodes and are not too
dissimilar to a human brain in that they derive their ``intelligence''
from the structure of their ``neurons'' and the strength of the
``synapses'' (edges) between these nodes. For this project, an ANN with
one input layer with the size of the number of features in the
underlying dataset, one hidden layer with ten nodes and an output layer
of the size of one was used to solve the CFB player classification
problem. The main advantages of ANN include its ability to model
complex, non-linear relationships between input and output variables,
its ability to infer generalized relationships on data it has never seen
before, its robustness towards local minima in the cost function as well
as its universality property, meaning it can fit any problem, provided
it is sufficiently large.

\hypertarget{ann-setup-and-backpropagation}{%
\section{2. ANN Setup and
Backpropagation}\label{ann-setup-and-backpropagation}}

The feedforward ANN used in this project is set up with three layers: an
input layer of the size of the number of features in the data, one
hidden layer consisting of ten nodes and an output layer containing a
single node. Having one hidden layer was deemed to be sufficient due to
the complexity of the underlying problem and the rule of thumb commonly
used to determine the size of the hidden layer is to take the mean
between the number of nodes in the input layer and the output layer. The
output layer was constructed to have one single node because the
classification problem at hand only determines between levels of one
class (drafted or not). A more generalized approach to ANN setup with
respects to the hidden layer would be to go through the process of
pruning, meaning starting out with a number of nodes that is rather too
large for the underlying problem and then subsequently eliminating
acitvations with corresponding weights close to zero.

In order to train the ANN, a process called backpropagation is applied.
This refers to the act of feeding labeled data into the network,
analyzing the cost (squared difference between the desired and actual
output) of the current combination of biases \(b_{1},...,b_{i}\),
activations \(a_{1},...,a_{i}\) and weights \(w_{1},...,w_{i}\) and
adjusting according to their influence on any given node in the
direction of greatest descent in the cost function \(\nabla_{a}C\).
Every acitvation \(a_L\) of layer \(L\) is given by the activation
function \(\sigma'(z_{L})\) with \(z_{L}=w_{L}a_{L-1}+b_{L}\). In this
project a logistic function was chosen as the activation function:

\[\sigma'(z)=\frac{1}{1+e^{-z}}\]

Subsequently, all changing effects for all neurons in the network are
added together and propagated backwards through the network, hence the
name backpropagation. This is repeated for all instances of training
data resulting in an average change over all training data which is
proportional to the negative gradient of the cost function:

\[\delta_L=\nabla_aC\odot\sigma'(z_L)\]

Because every activation is dependent on the bias \(b\) and weight \(w\)
as well as the connected activations from the previous layer, the total
change in the cost function is also dependent on the \(z\) which sums up
these values for the previous layer (the layer index \(L\) is given as a
superscript):

\[\frac{\partial C_0}{\partial w^{L}}=\frac{\partial z^L}{\partial w^L}\frac{\partial a^L}{\partial z^L}\frac{\partial C_0^L}{\partial a^L}\]

When combined with the sensitivity of the cost function to changes in
bias \(\frac{\partial C_0}{\partial b^L}\), this yields the combined
change to be made (the previous layer is indexed by \(k\) and the
current layer is indexed by \(j\)):

\[\frac{\partial C}{\partial w_{jk}^l}=a_{k}^{l-1}\delta_j^l\,\, and \,\,\frac{\partial C}{\partial b_j^l}=\delta_j^l, \,\, with\,\,\delta^l=((w^{l+1})^T\delta^{l+1})\odot\sigma'(z^l)\]

\hypertarget{data-standardization}{%
\section{3. Data Standardization}\label{data-standardization}}

Because an ANN posesses the ability to infer generalized relationships
on the training data without imposing any limitations on distribution
etc., it is not sensitive to differently scaled or centered features and
therefore does not change depending on monotonic transformation of the
input data. This means that there is no need to standardize the data
used, event though it is differently scaled across features (see
\href{https://github.com/NicSchuler/DSF_NFLDraftPrediction/tree/master/Data/READMEs}{RM\_DataHandling})

\hypertarget{tuning-the-model}{%
\section{4. Tuning the model}\label{tuning-the-model}}

As previously described, the ANN model could be tuned in terms of number
of hidden layers and especially on the number of nodes in the hidden
layer(s). Due to time and ressource constraints of this project, this is
not done in the ANN script, which leaves room for future improvement.
For the case at hand it is simply assumed to be sufficient to have one
hidden layer and the number of hidden nodes is determined by the general
rule of thumb descriped above. As the topic of ANNs is exceedingly deep,
there are a multitude of optimization functions available to optimize
ANN performance such as the Beale function, which, again due to project
constraints, have been neglected in this project.

\hypertarget{implementation-in-r}{%
\section{5. Implementation in R}\label{implementation-in-r}}

The application of ANN in R is explained below. All code is included in
the
\href{https://github.com/NicSchuler/DSF_NFLDraftPrediction/blob/master/Project_Scripts/ANN.R}{ANN
script} and step-by-step comments are provided.

\hypertarget{training-the-ann-model}{%
\subsection{5.1 Training the ANN Model}\label{training-the-ann-model}}

For training the data from years 2007 to 2013 of all unsampled and
sampled datasets are used respectively. The ANN model is trained using
the \texttt{nlminb()} function of the \texttt{h2o} package which
implements quasi-Newton optimization methods. These are used in order to
reduce computational power that would be needed for full Newton's
methods in order to find the twice differentiable global minimum of a
complex non-linear function.

\begin{Shaded}
\begin{Highlighting}[]
\NormalTok{BP_pred <-}\StringTok{ }\KeywordTok{nlminb}\NormalTok{(}\DataTypeTok{start =}\NormalTok{ ANN_par,}
                 \DataTypeTok{objective =}\NormalTok{ ANN_cost, }
                 \DataTypeTok{gradient =}\NormalTok{ ANN_grad,}
                 \DataTypeTok{hessian =} \OtherTok{NULL}\NormalTok{,}
                 \DataTypeTok{L_i_size =}\NormalTok{ L_i_size,}
                 \DataTypeTok{L_h_size =}\NormalTok{ L_h_size,}
                 \DataTypeTok{L_o_size =}\NormalTok{ L_o_size,}
                 \DataTypeTok{x =}\NormalTok{ x, }\DataTypeTok{y =}\NormalTok{ y,}
                 \DataTypeTok{lambda =} \DecValTok{1}\NormalTok{,}
                 \DataTypeTok{control =}\NormalTok{ options)}
\end{Highlighting}
\end{Shaded}

As the \texttt{start} parameter, the combined thetas containing only 1s
is passed. The \texttt{ANN\_cost} and \texttt{ANN\_grad} functions have
been previously defined as the cost and gradient function to optimize.
The \texttt{L\_i\_size}, \texttt{L\_h\_size} and \texttt{L\_o\_size}
arguments define the dimensions of the input, hidden and output layers
as described above.

The resulting vetor of the two thetas is then saved to be used for
prediction on unlabeled testing data later on:

\begin{Shaded}
\begin{Highlighting}[]
\NormalTok{BP_par <-}\StringTok{ }\NormalTok{BP_pred}\OperatorTok{$}\NormalTok{par}
\end{Highlighting}
\end{Shaded}

\hypertarget{variable-importance}{%
\subsection{5.2 Variable Importance}\label{variable-importance}}

Largely due to the universality property of ANNs, variable importance
and the construction of hypermodels including preliminary methods for
variable selection have not been implemented. This is following the
logic that an ANN should be capable of dealing with a mixture of
features of differing importance by assigning very small weights to the
corresponding nodes. Because it is not clear if the neural network is
tuned optimally, it is not possible to give statements about the
universality of the specific ANN used in this project. Because there is
a possibility that the underlying ANN is oversimplified in respects to
the problem at hand, this leaves room for future improvement.

\hypertarget{model-performance}{%
\subsection{5.3 Model Performance}\label{model-performance}}

In order to gain an oversight of the model performance, the ANN models
are tested on their training (using the years 2007 through 2013) and
testing (using the year 2014) fit for all positions and all sampling
methods. Both are based on unsampled data, allowing for the model's
predictive ability on imbalanced class sizes, as they occur in the
underlying business case, to be analyzed. The training and testing data
performance can be viewed below (also see the
\href{https://github.com/NicSchuler/DSF_NFLDraftPrediction/blob/master/Data/READMEs/RM_PerformanceMeasurement.pdf}{between
model comparison}):

\begin{longtable}[]{@{}llrrrrrrrrrrrrrrrr@{}}
\caption{training data performance}\tabularnewline
\toprule
Method & Sampling & QB\_TP & QB\_TN & QB\_FP & QB\_FN & WR\_TP & WR\_TN
& WR\_FP & WR\_FN & RB\_TP & RB\_TN & RB\_FP & RB\_FN & Together\_TP &
Together\_TN & Together\_FP & Together\_FN\tabularnewline
\midrule
\endfirsthead
\toprule
Method & Sampling & QB\_TP & QB\_TN & QB\_FP & QB\_FN & WR\_TP & WR\_TN
& WR\_FP & WR\_FN & RB\_TP & RB\_TN & RB\_FP & RB\_FN & Together\_TP &
Together\_TN & Together\_FP & Together\_FN\tabularnewline
\midrule
\endhead
ANN & no\_sampling & 32 & 316 & 41 & 21 & 81 & 1096 & 83 & 40 & 38 & 517
& 52 & 21 & 140 & 1924 & 187 & 87\tabularnewline
ANN & oversampling & 73 & 180 & 0 & 157 & 159 & 768 & 5 & 368 & 84 & 344
& 6 & 194 & 313 & 1269 & 14 & 742\tabularnewline
ANN & undersampling & 73 & 176 & 0 & 161 & 158 & 761 & 6 & 375 & 83 &
347 & 7 & 191 & 311 & 1301 & 16 & 710\tabularnewline
ANN & Rose\_both & 73 & 172 & 0 & 165 & 159 & 746 & 5 & 390 & 84 & 351 &
6 & 187 & 314 & 1263 & 13 & 748\tabularnewline
ANN & Smote & 72 & 199 & 1 & 138 & 162 & 708 & 2 & 428 & 84 & 334 & 6 &
204 & 314 & 1229 & 13 & 782\tabularnewline
\bottomrule
\end{longtable}

\begin{longtable}[]{@{}llrrrrrrrrrrrrrrrr@{}}
\caption{testing data performance}\tabularnewline
\toprule
Method & Sampling & QB\_TP & QB\_TN & QB\_FP & QB\_FN & WR\_TP & WR\_TN
& WR\_FP & WR\_FN & RB\_TP & RB\_TN & RB\_FP & RB\_FN & Together\_TP &
Together\_TN & Together\_FP & Together\_FN\tabularnewline
\midrule
\endfirsthead
\toprule
Method & Sampling & QB\_TP & QB\_TN & QB\_FP & QB\_FN & WR\_TP & WR\_TN
& WR\_FP & WR\_FN & RB\_TP & RB\_TN & RB\_FP & RB\_FN & Together\_TP &
Together\_TN & Together\_FP & Together\_FN\tabularnewline
\midrule
\endhead
ANN & no\_sampling & 8 & 98 & 3 & 7 & 11 & 312 & 11 & 24 & 9 & 171 & 5 &
11 & 27 & 584 & 20 & 39\tabularnewline
ANN & oversampling & 10 & 56 & 1 & 49 & 22 & 216 & 0 & 120 & 12 & 107 &
2 & 75 & 44 & 372 & 3 & 251\tabularnewline
ANN & undersampling & 11 & 52 & 0 & 53 & 22 & 213 & 0 & 123 & 12 & 107 &
2 & 75 & 44 & 386 & 3 & 237\tabularnewline
ANN & Rose\_both & 11 & 55 & 0 & 50 & 21 & 211 & 1 & 125 & 12 & 105 & 2
& 77 & 44 & 377 & 3 & 246\tabularnewline
ANN & Smote & 10 & 58 & 1 & 47 & 22 & 190 & 0 & 146 & 12 & 103 & 2 & 79
& 44 & 358 & 3 & 265\tabularnewline
\bottomrule
\end{longtable}


\end{document}
