\documentclass[]{article}
\usepackage{lmodern}
\usepackage{amssymb,amsmath}
\usepackage{ifxetex,ifluatex}
\usepackage{fixltx2e} % provides \textsubscript
\ifnum 0\ifxetex 1\fi\ifluatex 1\fi=0 % if pdftex
  \usepackage[T1]{fontenc}
  \usepackage[utf8]{inputenc}
\else % if luatex or xelatex
  \ifxetex
    \usepackage{mathspec}
  \else
    \usepackage{fontspec}
  \fi
  \defaultfontfeatures{Ligatures=TeX,Scale=MatchLowercase}
\fi
% use upquote if available, for straight quotes in verbatim environments
\IfFileExists{upquote.sty}{\usepackage{upquote}}{}
% use microtype if available
\IfFileExists{microtype.sty}{%
\usepackage{microtype}
\UseMicrotypeSet[protrusion]{basicmath} % disable protrusion for tt fonts
}{}
\usepackage[margin=1in]{geometry}
\usepackage{hyperref}
\PassOptionsToPackage{usenames,dvipsnames}{color} % color is loaded by hyperref
\hypersetup{unicode=true,
            pdftitle={README for the performance measurement part},
            pdfauthor={Group 2},
            colorlinks=true,
            linkcolor=Maroon,
            citecolor=Blue,
            urlcolor=blue,
            breaklinks=true}
\urlstyle{same}  % don't use monospace font for urls
\usepackage{longtable,booktabs}
\usepackage{graphicx,grffile}
\makeatletter
\def\maxwidth{\ifdim\Gin@nat@width>\linewidth\linewidth\else\Gin@nat@width\fi}
\def\maxheight{\ifdim\Gin@nat@height>\textheight\textheight\else\Gin@nat@height\fi}
\makeatother
% Scale images if necessary, so that they will not overflow the page
% margins by default, and it is still possible to overwrite the defaults
% using explicit options in \includegraphics[width, height, ...]{}
\setkeys{Gin}{width=\maxwidth,height=\maxheight,keepaspectratio}
\IfFileExists{parskip.sty}{%
\usepackage{parskip}
}{% else
\setlength{\parindent}{0pt}
\setlength{\parskip}{6pt plus 2pt minus 1pt}
}
\setlength{\emergencystretch}{3em}  % prevent overfull lines
\providecommand{\tightlist}{%
  \setlength{\itemsep}{0pt}\setlength{\parskip}{0pt}}
\setcounter{secnumdepth}{0}
% Redefines (sub)paragraphs to behave more like sections
\ifx\paragraph\undefined\else
\let\oldparagraph\paragraph
\renewcommand{\paragraph}[1]{\oldparagraph{#1}\mbox{}}
\fi
\ifx\subparagraph\undefined\else
\let\oldsubparagraph\subparagraph
\renewcommand{\subparagraph}[1]{\oldsubparagraph{#1}\mbox{}}
\fi

%%% Use protect on footnotes to avoid problems with footnotes in titles
\let\rmarkdownfootnote\footnote%
\def\footnote{\protect\rmarkdownfootnote}

%%% Change title format to be more compact
\usepackage{titling}

% Create subtitle command for use in maketitle
\providecommand{\subtitle}[1]{
  \posttitle{
    \begin{center}\large#1\end{center}
    }
}

\setlength{\droptitle}{-2em}

  \title{README for the performance measurement part}
    \pretitle{\vspace{\droptitle}\centering\huge}
  \posttitle{\par}
    \author{Group 2}
    \preauthor{\centering\large\emph}
  \postauthor{\par}
      \predate{\centering\large\emph}
  \postdate{\par}
    \date{\texttt{December,\ 2nd,\ 2019}}


\begin{document}
\maketitle
\begin{abstract}
This Chapter is describes the evaluation of the best sampling and model
combination, which is computed in the script `PerformanceMeasurement.R'.
\end{abstract}

\hypertarget{the-goal-of-our-performance-measurement}{%
\subsection{The goal of our performance
measurement}\label{the-goal-of-our-performance-measurement}}

After training a model it is important to check how well it performes.
Since we not only trained one model but 140 in total (7 algos * 5
datasets * 4 positions), it is even more important to compare all the
different models to see which one is the best. For our business case,
this means that we want to see which combination of method and sampling
we should take to predict which position, or if we even have a model,
which does it better when we leave them together.

\hypertarget{how-to-compare-the-methodsampling-combinations-for-all-positions}{%
\subsection{How to compare the method/sampling combinations for all
positions}\label{how-to-compare-the-methodsampling-combinations-for-all-positions}}

After training the models with 10-fold cross-validation, we applied them
on all the (unsampled) data from 2007 to 2013 to obtain the true
positives, true negatives, false positives and false negatives. Since we
used the cross-validation for training, this will not give us the
training error, but the testing error we get with all the available data
prior to the 2014 NFL Draft. At the end of every model-script, we save
this information separately, to bring it all together in the script
`PerformanceMeasurement.R'.

The first step, is to bring them all into one dataframe, to use them
easier. Then we make sure, that we used the same, unsampled data by
computing the sums of TP, TN, FP, FN, for every method/sampling/position
combination. In the CheckTibble we now see, that for every position the
whole column contains always the same number, which means this is the
case.

Then we calculate the ratio of correct classification, which is equal
to:

\[\frac{\text{Correct Classifications}}{\text{All Classifications}} = \frac{TP + TN}{TP+TP+FP+FN}\]

The result is a table showing the percentage of correct classification
for all the 140 model/sampling/position combination. This table is quite
big, but it is still interesting to have a look at it, to see how well
which combination performes.

\begin{longtable}[]{@{}llrrrr@{}}
\caption{Ratio of correct classification}\tabularnewline
\toprule
Method & Sampling & QB & WR & RB & Together\tabularnewline
\midrule
\endfirsthead
\toprule
Method & Sampling & QB & WR & RB & Together\tabularnewline
\midrule
\endhead
ClassificationTree & no\_sampling & 0.9098 & 0.9208 & 0.9156 &
0.8939\tabularnewline
ClassificationTree & oversampling & 0.8780 & 0.8415 & 0.8631 &
0.7930\tabularnewline
ClassificationTree & undersampling & 0.7098 & 0.8254 & 0.8137 &
0.7938\tabularnewline
ClassificationTree & Rose\_both & 0.8317 & 0.7985 & 0.8057 &
0.7746\tabularnewline
ClassificationTree & Smote & 0.8439 & 0.8062 & 0.7850 &
0.7998\tabularnewline
KNN & no\_sampling & 0.8488 & 0.8892 & 0.8646 & 0.8755\tabularnewline
KNN & oversampling & 0.8366 & 0.8915 & 0.8519 & 0.8725\tabularnewline
KNN & undersampling & 0.6463 & 0.8331 & 0.7962 & 0.7994\tabularnewline
KNN & Rose\_both & 0.7317 & 0.8192 & 0.7834 & 0.7981\tabularnewline
KNN & Smote & 0.8610 & 0.9023 & 0.8726 & 0.8918\tabularnewline
NaiveBayes & no\_sampling & 0.8268 & 0.8908 & 0.8758 &
0.8901\tabularnewline
NaiveBayes & oversampling & 0.7000 & 0.7746 & 0.8137 &
0.7998\tabularnewline
NaiveBayes & undersampling & 0.6878 & 0.7946 & 0.8328 &
0.7968\tabularnewline
NaiveBayes & Rose\_both & 0.7341 & 0.7231 & 0.8615 &
0.7930\tabularnewline
NaiveBayes & Smote & 0.7317 & 0.8077 & 0.7213 & 0.8798\tabularnewline
randomForest & no\_sampling & 1.0000 & 0.9977 & 1.0000 &
0.9979\tabularnewline
randomForest & oversampling & 1.0000 & 0.9985 & 1.0000 &
0.9996\tabularnewline
randomForest & undersampling & 0.7415 & 0.8838 & 0.8344 &
0.8499\tabularnewline
randomForest & Rose\_both & 0.9073 & 0.9462 & 0.9395 &
0.9371\tabularnewline
randomForest & Smote & 1.0000 & 0.9962 & 1.0000 & 0.9983\tabularnewline
ANN & no\_sampling & 0.8488 & 0.9054 & 0.8838 & 0.8828\tabularnewline
ANN & oversampling & 0.8593 & 0.8664 & 0.8574 & 0.8531\tabularnewline
ANN & undersampling & 0.8559 & 0.8771 & 0.8708 & 0.8440\tabularnewline
ANN & Rose\_both & 0.8923 & 0.8570 & 0.8558 & 0.8512\tabularnewline
ANN & Smote & 0.8639 & 0.8853 & 0.8692 & 0.8628\tabularnewline
LogisticRegression & no\_sampling & 0.8510 & 0.9072 & 0.8938 &
0.8980\tabularnewline
LogisticRegression & oversampling & 0.7876 & 0.8608 & 0.8246 &
0.8432\tabularnewline
LogisticRegression & undersampling & 0.7678 & 0.8637 & 0.8167 &
0.8432\tabularnewline
LogisticRegression & Rose\_both & 0.7879 & 0.8606 & 0.8248 &
0.8451\tabularnewline
LogisticRegression & Smote & NA & NA & NA & NA\tabularnewline
\bottomrule
\end{longtable}

We can see these two tendencies in the models:

\begin{itemize}
\tightlist
\item
  Better performance with unsampled data, than with sampled data
\item
  Better performance when we split the positions manually
\end{itemize}

\hypertarget{our-best-models}{%
\subsection{Our best models}\label{our-best-models}}

With regards to the business case, we now need to decide which
model/models we will use to predict the 2014 NFL Draft. Here you can see
the best model/sampling combination for all positions.

\begin{longtable}[]{@{}lllr@{}}
\caption{The best model/sampling combinations by
position}\tabularnewline
\toprule
Position & Method & Sampling & Accuracy\tabularnewline
\midrule
\endfirsthead
\toprule
Position & Method & Sampling & Accuracy\tabularnewline
\midrule
\endhead
QB & randomForest & no\_sampling & 1.0000\tabularnewline
WR & randomForest & oversampling & 0.9985\tabularnewline
RB & randomForest & no\_sampling & 1.0000\tabularnewline
Together & randomForest & oversampling & 0.9996\tabularnewline
\bottomrule
\end{longtable}

\hypertarget{discussion}{%
\subsection{Discussion}\label{discussion}}

The great and sometimes perfect results of the random forest model show
a high risk of a massive overfit. The other models with accuracies
between 89\% and 93\% seem to be very good at the first sight. But we
have to keep the reason for filtering out the players with \textless10
played games and sampling the data in mind. Here we applied the models
to the unsampled but filtered data, which only contains 12.4\% of
drafted players. This means, that a model predicting ``not drafted'' for
every player would not perform much worse, since it would still have an
accuracy of 87.6\%.

Interpretations of models are always quite difficult, but the following
thoughts are pretty likely to be true, according to the TP/TN/FP/FN. Our
models are pretty good at predicting the likelihood of players being
drafted that didn't perform very well in college football being low. It
probably also does not too bad in predicting great players to be
drafted, that nearly must be picked (and probably are picked early in
the draft). But there is probably much room for improvement for all the
players that still performed well in college, and might or might not be
drafted.

We would like to close the circle to one of our fist lessons in machine
learning, in which we were taught the following very high level formula
for models (the right part). Y denotes the true outcome (in our case
whether a player is drafted or not), f(X) is the true pattern that
describes it, and \(\epsilon\) is the noise, which appears to be random.
With our models (the left part) we try to predict a \(\hat{Y}\), which
shall be as close as possible to the true Y.

\[\hat{f}(X) = \hat{Y} \approx Y = f(X) + \epsilon\] Looking at this
inequation we can think of three possibilities, why our models make so
many mistakes:

\begin{itemize}
\tightlist
\item
  Our models \(\hat{f_i}(X)\) do not include enough variables and/or are
  not sophisticated enough
\item
  The data is not good enough
\item
  The NFL draft contains a pretty large \(\epsilon\)
\end{itemize}


\end{document}
